\documentclass[11pt,a4paper,titlepage]{article}

\usepackage[utf8]{inputenc}

%%% PAGE DIMENSIONS
\usepackage{geometry} % to change the page dimensions
\geometry{a4paper} % or letterpaper (US) or a5paper or....
% \geometry{margin=2in} % for example, change the margins to 2 inches all round

\usepackage{graphicx} % support the \includegraphics command and options

\usepackage[parfill]{parskip} % Activate to begin paragraphs with an empty line rather than an indent

%%% PACKAGES
\usepackage{booktabs} % for much better looking tables
\usepackage{array} % for better arrays (eg matrices) in maths
\usepackage{paralist} % very flexible & customisable lists (eg. enumerate/itemize, etc.)
\usepackage{verbatim} % adds environment for commenting out blocks of text & for better verbatim
\usepackage{lipsum}
\usepackage{caption,subcaption}
\usepackage{amsmath}
% These packages are all incorporated in the memoir class to one degree or another...

%%% HEADERS & FOOTERS
\usepackage{fancyhdr} % This should be set AFTER setting up the page geometry
\setlength{\headheight}{13.6pt}
\pagestyle{fancy} % options: empty , plain , fancy
\renewcommand{\headrulewidth}{0pt} % customise the layout...
\lhead{2IMV20 Visualisation}\chead{}\rhead{Eindhoven University of Technology}
\lfoot{}\cfoot{\thepage}\rfoot{}


%%% END Article customizations

\begin{document}

\begin{titlepage} % Suppresses displaying the page number on the title page and the subsequent page counts as page 1
	\newcommand{\HRule}{\rule{\linewidth}{0.5mm}} % Defines a new command for horizontal lines, change thickness here
	
	%------------------------------------------------
	%	Headings
	%------------------------------------------------
	
	\textsc{\LARGE Eindhoven University of Technology}\\[1.5cm] % Main heading such as the name of your university/college
	
	\textsc{\Large 2IMV20}\\[0.5cm] % Major heading such as course name
	
	\textsc{\large Visualisation}\\[0.5cm] % Minor heading such as course title
	
	%------------------------------------------------
	%	Title
	%------------------------------------------------
	
	\HRule\\[0.4cm]
	
	{\huge\bfseries Visualising Immigration Patterns in the Netherlands}\\[0.4cm] % Title of your document
	
	\HRule\\[1.5cm]
	
	%------------------------------------------------
	%	Author(s)
	%------------------------------------------------
	
	{\large\textit{Author}}\\
	Stanley \textsc{Clark} % Your name
	
	%------------------------------------------------
	%	Date
	%------------------------------------------------
	
	\vfill\vfill\vfill % Position the date 3/4 down the remaining page
	
	{\large\today} % Date, change the \today to a set date if you want to be precise
	
	%----------------------------------------------------------------------------------------
	
	\vfill % Push the date up 1/4 of the remaining page
	
\end{titlepage}

\tableofcontents
\clearpage

\setcounter{page}{1}

\section{Introduction}

Recent elections in the western world have seen what can only be described as an anti-immigrant sentiment that is protruding its teeth at all who oppose it. Indeed, also in the Netherlands one can see the rise of parties that advertise a curbing of the migrant flow and rethinking our policies on asylum as well as visa restrictions. A disadvantage of relying on news media and political figures for accurate information is the lack of readily available sources on the matter at one's disposal. Think of when was the last time that you saw a newspaper article stating a claim, and then failing to provide any links to where those particular numbers came from. Thus is the motivation for this paper. In it, we will attempt to show how one can use publicly available data on immigration figures to visualise information in a user friendly tool, that allows for exploration and the gathering of your own insights with regards to the immigration debate.

\section{Dataset Selection}


\end{document}
